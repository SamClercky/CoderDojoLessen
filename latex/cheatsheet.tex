\chapter*{Cheatsheet}%

\section{HTML}%
\label{sec:html}

\subsection{Structuur}%
\label{sub:structuur}

Elk \HTML bestand begint met volgende tekst. Dit vertelt de computer dat we \HTML willen gebruiken en geeft extra informatie over hoe de computer onze website moet weergeven.
\inputminted{html}{../cheatsheet/basis_html.html}

\HTML bestaat uit tags. Een tag komt in 2 vormen voor:
\begin{itemize}
    \item Met een begin en een eind die inhoud bevat:
        \begin{minted}{html}
<div>inhoud van de tag</div>
        \end{minted}
    \item Zonder inhoud:
        \begin{minted}{html}
<input />
        \end{minted}
\end{itemize}

Elke tag kan ook extra informatie bevatten via attributen. Deze zijn vaak optioneel maar handig.

\begin{itemize}
    \item Met een begin en een eind die inhoud bevat:
        \begin{minted}{html}
<div id="mijn_id" class="mijn_klasse">
    inhoud van de tag
</div>
        \end{minted}
    \item Zonder inhoud:
        \begin{minted}{html}
<input id="mijn_id" type="password" />
        \end{minted}
\end{itemize}

\Opm \HTML houd geen rekening met spaties, tabs, ... Dus volgende 2 notaties zijn hetzelfde:
\begin{minted}{html}
<div
    class="blabla"
    id="blabla">
        Dit is indoud
</div>
<div class="blabla" id="blabla">Dit is indoud</div>
<div class="blabla" id="blabla">Dit is indoud met veel spaties
                        </div>
\end{minted}

\subsection{Voorbeeld rood vierkant}%
\label{sub:voorbeeld_rood_vierkant}

Plaatsen van een rood vierkant op een gekozen locatie:
\inputminted{html}{../cheatsheet/pos_div.html}

Het grote verschil met de begincode is de \textbf{div}. En deze heeft 1 groot attribuut \textbf{style}. De waarde hierin is eigenlijk \CSS waarover later meer.

Kort overlopen van de verschillende waarden in de \textbf{style}:
\begin{itemize}
    \item \emph{background}: achtergrond rood (red) maken
    \item \emph{width}: breedte op \emph{100px} zetten
    \item \emph{height}: hoogte op \emph{100px} zetten
    \item \emph{position}: nodig om een niet standaard positie te kunnen opgeven
    \item \emph{left}: afstand van de linker rand
    \item \emph{right}: afstand van de top
\end{itemize}

\TODO foto van gerenderde website

\subsection{Belangrijke HTML tags}%
\label{sub:belangrijke_html_tags}

\begin{itemize}
    \item \emph{html}: Bevat alle andere \HTML tags
    \item \emph{head}: Normaal nooit zichtbaar, maar bevat belangrijke informatie over de \HTML
    \item \emph{body}: Alles in deze tag is zichtbaar op het scherm
    \item \emph{title}: De titel van je website en komt enkel voor in een \emph{head}-tag
    \item \emph{div}: Een block-container zonder extra betekenis
    \item \emph{img}: Voeg een afbeelding toe met de \emph{src}-attribuut (url)
    \item \emph{input}: Tekstveld, handig voor vragen van een spelersnaam
    \item \emph{style}: \CSS voor gans de webpagina
    \item \emph{script}: \JS voor de webpagina
\end{itemize}

\subsection{Belangrijke attributen}%
\label{sub:belangrijke_attributen}

\begin{itemize}
    \item \emph{style}: Voeg \CSS toe direct in de \HTML
    \item \emph{id}: Speciale naam voor de tag (moet uniek zijn voor elke tag)
    \item \emph{class}: Zelfde als \emph{id} maar moet niet uniek zijn
    \item \emph{src} (enkel \emph{img}): De url van de tonen afbeelding
    \item \emph{value} (enkel \emph{input}): De ingevulde waarde in het tekstveld
    \item \emph{type} (enkel \emph{input}): Soort input. Belangrijke mogelijke waarden:
        \begin{itemize}
            \item \emph{text} (Standaard waarde): normale tekst
            \item \emph{password}: wachtwoord
            \item \emph{number}: getal
        \end{itemize}
\end{itemize}

\section{CSS}%
\label{sec:css}

\subsection{Structuur}%
\label{sub:structuur}

\begin{minted}{css}
selector {
    eigenschap: waarde;
}
\end{minted}

Er zijn 2 manieren om \CSS toe te voegen. Ofwel via een \emph{style}-tag in de \emph{head}-tag:
\begin{minted}{html}
...
<head>
    ...
    <style>
        selector1 {
            eigenschap: waarde;
        }
        selector2 {
            eigenschap: waarde;
        }
        selector3 {
            eigenschap: waarde;
        }
    </style>
</head>
...
\end{minted}

Een andere manier is zoals eerder met de \emph{style}-attributen. Het grote verschil is dat er nu geen nood meer is aan de selector:
\begin{minted}{html}
...
<div style="
    eigenschap1: waarde;
    eigenschap2: waarde;
    eigenschap3: waarde;
" ></div>
...
\end{minted}

Een selector is een manier om een aantal tags te selecteren en daarvan de
eigenschappen te wijzigen. Er zijn 3 grote manieren van selecteren:
\begin{itemize}
    \item Via tagnaam: bv: \textbf{div}: selecteert alle divs.
    \item Via classmaam: bv: \textbf{.klasse}: selecteert alle tags die
        \mintinline{html}|class="klasse"| hebben
    \item Via id: bv: \mintinline{css}|#id_naam|: selecteert de tag die
        \mintinline{html}|id="id_naam"| heeft. Dit is normaal altijd 1 tag.
\end{itemize}
