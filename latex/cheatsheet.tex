\chapter{Cheatsheet}%

\section{HTML}%
\label{sec:html}

\subsection{Structuur}%
\label{sub:structuur}

Elk \HTML bestand begint met volgende tekst. Dit vertelt de computer dat we \HTML willen gebruiken en geeft extra informatie over hoe de computer onze website moet weergeven.
\inputminted{html}{../cheatsheet/basis_html.html}

\HTML bestaat uit tags. Een tag komt in 2 vormen voor:
\begin{itemize}
    \item Met een begin en een eind die inhoud bevat:
        \begin{minted}{html}
<div>inhoud van de tag</div>
        \end{minted}
    \item Zonder inhoud:
        \begin{minted}{html}
<input />
        \end{minted}
\end{itemize}

Elke tag kan ook extra informatie bevatten via attributen. Deze zijn vaak optioneel maar handig.

\begin{itemize}
    \item Met een begin en een eind die inhoud bevat:
        \begin{minted}{html}
<div id="mijn_id" class="mijn_klasse">
    inhoud van de tag
</div>
        \end{minted}
    \item Zonder inhoud:
        \begin{minted}{html}
<input id="mijn_id" type="password" />
        \end{minted}
\end{itemize}

\Opm \HTML houd geen rekening met spaties, tabs, ... Dus volgende 2 notaties zijn hetzelfde:
\begin{minted}{html}
<div
    class="blabla"
    id="blabla">
        Dit is indoud
</div>
<div class="blabla" id="blabla">Dit is indoud</div>
<div class="blabla" id="blabla">Dit is indoud met veel spaties
                        </div>
\end{minted}

\subsection{Voorbeeld rood vierkant}%
\label{sub:voorbeeld_rood_vierkant}

Plaatsen van een rood vierkant op een gekozen locatie:
\inputminted{html}{../cheatsheet/pos_div.html}

Het grote verschil met de begincode is de \textbf{div}. En deze heeft 1 groot attribuut \textbf{style}. De waarde hierin is eigenlijk \CSS waarover later meer.

Kort overlopen van de verschillende waarden in de \textbf{style}:
\begin{itemize}
    \item \emph{background}: achtergrond rood (red) maken
    \item \emph{width}: breedte op \emph{100px} zetten
    \item \emph{height}: hoogte op \emph{100px} zetten
    \item \emph{position}: nodig om een niet standaard positie te kunnen opgeven
    \item \emph{left}: afstand van de linker rand
    \item \emph{right}: afstand van de top
\end{itemize}

\TODO foto van gerenderde website

\subsection{Belangrijke HTML tags}%
\label{sub:belangrijke_html_tags}

\begin{itemize}
    \item \emph{html}: Bevat alle andere \HTML tags
    \item \emph{head}: Normaal nooit zichtbaar, maar bevat belangrijke informatie over de \HTML
    \item \emph{body}: Alles in deze tag is zichtbaar op het scherm
    \item \emph{title}: De titel van je website en komt enkel voor in een \emph{head}-tag
    \item \emph{div}: Een block-container zonder extra betekenis
    \item \emph{img}: Voeg een afbeelding toe met de \emph{src}-attribuut (url)
    \item \emph{input}: Tekstveld, handig voor vragen van een spelersnaam
    \item \emph{style}: \CSS voor gans de webpagina
    \item \emph{script}: \JS voor de webpagina
\end{itemize}

\subsection{Belangrijke attributen}%
\label{sub:belangrijke_attributen}

\begin{itemize}
    \item \emph{style}: Voeg \CSS toe direct in de \HTML
    \item \emph{id}: Speciale naam voor de tag (moet uniek zijn voor elke tag)
    \item \emph{class}: Zelfde als \emph{id} maar moet niet uniek zijn
    \item \emph{src} (enkel \emph{img}): De url van de tonen afbeelding
    \item \emph{value} (enkel \emph{input}): De ingevulde waarde in het tekstveld
    \item \emph{type} (enkel \emph{input}): Soort input. Belangrijke mogelijke waarden:
        \begin{itemize}
            \item \emph{text} (Standaard waarde): normale tekst
            \item \emph{password}: wachtwoord
            \item \emph{number}: getal
        \end{itemize}
\end{itemize}

\section{CSS}%
\label{sec:css}

\subsection{Structuur}%
\label{sub:structuur}

\begin{minted}{css}
selector {
    eigenschap: waarde;
}
\end{minted}

Er zijn 2 manieren om \CSS toe te voegen. Ofwel via een \emph{style}-tag in de \emph{head}-tag:
\begin{minted}{html}
...
<head>
    ...
    <style>
        selector1 {
            eigenschap: waarde;
        }
        selector2 {
            eigenschap: waarde;
        }
        selector3 {
            eigenschap: waarde;
        }
    </style>
</head>
...
\end{minted}

Een andere manier is zoals eerder met de \emph{style}-attributen. Het grote verschil is dat er nu geen nood meer is aan de selector:
\begin{minted}{html}
...
<div style="
    eigenschap1: waarde;
    eigenschap2: waarde;
    eigenschap3: waarde;
" ></div>
...
\end{minted}

Een selector is een manier om een aantal tags te selecteren en daarvan de
eigenschappen te wijzigen. Er zijn 3 grote manieren van selecteren:
\begin{itemize}
    \item Via tagnaam: bv: \textbf{div}: selecteert alle divs.
    \item Via classmaam: bv: \textbf{.klasse}: selecteert alle tags die
        \mintinline{html}|class="klasse"| hebben
    \item Via id: bv: \mintinline{css}|#id_naam|: selecteert de tag die
        \mintinline{html}|id="id_naam"| heeft. Dit is normaal altijd 1 tag.
\end{itemize}

\subsection{Interessante eigenschappen}%
\label{sub:interessante_eigenschappen}

Hier is een lijstje van de meest interessante eigenschappen voor het opmaken met
CSS. Alle eigenschappen veranderen iets van alleen de geselecteerde elementen!
\begin{itemize}
    \item \textbf{background}: bepaalt de achtergrond
    \item \textbf{color}: bepaalt de tekstkleur
    \item \textbf{width}: bepaalt de breedte
    \item \textbf{height}: bepaalt de hoogte
    \item \textbf{position}: bepaalt de manier waarop de positie van het
        element wordt bepaalt. De belangrijkste zijn:
        \begin{itemize}
            \item \textbf{absolute}: positie ten opzichte van de linker bovenhoek
            \item \textbf{relative}: positie ten opzichte van de linker
                bovenhoek van de ouder (parent)
            \item \textbf{fixed}: zoals absolute maar wanneer je scrolt, blijft
                het element op zijn plaats
        \end{itemize}
    \item \textbf{left}: afstand van links
    \item \textbf{top}: afstand van de bovenkant
    \item \textbf{right}: afstand van rechts
    \item \textbf{bottom}: afstand van de onderkant
    \item \textbf{margin-\{left,top,right,bottom\}}: minimale afstand van
        links,boven,rechts,onder
    \item \textbf{padding-\{left,top,right,bottom\}}: minimale interne afstand
        van links,boven,rechts,onder
    \item \textbf{opacity}: doorzichtbaarheid
\end{itemize}

De waarden die je in bovenstaande eigenschappen kunt invullen, worden hieronder
opgelijst:
\begin{itemize}
    \item \textbf{Afstand}: px (pixels), \% (procent van de ouder)
    \item \textbf{Kleur}: 
        \begin{itemize}
            \item \textbf{Kleurnaam}: red (rood), green (groen), blue (blauw),
                black (zwart), white (wit)
            \item \textbf{rgb(0-255, 0-255, 0-255)}: Kleur samenstellen uit
                r(ood), g(roen) en b(lauw)
            \item \textbf{hsl(hue, saturation, lightness)}: Kleur samenstellen uit
                hue (tint), saturation (verzadiging), lightness (lichtheid)
            \item \textbf{\#XXXXXX}: Hetzelfde als rgb maar korter. Elke kleur 
                bestaat uit 2 tekens tussen 0-9+A-F. Enkele voorbeelden:
                \begin{itemize}
                    \item Wit: \#FFFFFF
                    \item Zwart: \#000000
                    \item Rood: \#FF0000
                    \item Groen: \#00FF00
                    \item Blauw: \#0000FF
                \end{itemize}
        \end{itemize}
\end{itemize}

\section{JavaScript}%
\label{sec:javascript}

\subsection{Structuur}%
\label{sub:structuur}

Om JavaScript de kunnen gebruiken, moet je deze in de HTML in een
\textbf{script}-tag zetten
\begin{minted}{html}
<script type="text/javascript">
    // Dit is een comment
    // Ik wordt nooit uitgevoerd, maar ben hier
    // om jou te helpen
    // In deze tag komt verder alle JavaScript
    console.log("Kijk ik ben cool, druk F12 om mij te zien!");
</script>
\end{minted}

De meest algemene structuur die te vinden is in JavaScript is het
\textbf{statement} 
\begin{minted}{javascript}
console.log("Een beetje veel tekst :)");
\end{minted}
Een aantal dingen om op te merken:
\begin{itemize}
    \item Elke statement bevat een actie (iets doen)
    \item Elke statement eindigt met een punt komma \textbf{;} 
\end{itemize}

Je kunt een statement zien als 1 blokje in Scratch en veel statement na elkaar
is alsof je veel blokjes aan elkaar kunt vastmaken.

\subsubsection{Variabelen}%
\label{ssub:Variabelen}

In Scratch kun je ook variabelen maken. Dit waren stukjes geheugen waar je een 
waarde aan kon meegeven en zo een score kon bijhouden. In JavaScript doe je dit
zo:
\begin{minted}{javascript}
let naamVariabele = 1;
naamVariabele = naamVariabele + 1;
\end{minted}
Een variabele aanmaken begint altijd met het woordje \textbf{let}. Hiermee zeg
je tegen je computer dat hij een stukje geheugen moet reserveren en dat de naam
\textbf{naamVariabele} moet geven.

\Opm Je kunt maar 1 keer een variabele aanvragen onder dezelfde naam in
dezelfde scope (hierover later meer)! Anders gaat de computer klagen. Het is
alsof je in een restaurant 2 reserveringen maakt onder  dezelfde naam. Als je
dan toekomt en ze vragen naar je naam, weten zij niet meer welke reservatie
precies voor wie was.

Nu je een variabele hebt, kun je iedere keer je de waarde die je erin hebt
opgeslagen, nodig hebt, gewoon de naam van de variabele gebruiken.

Met het \textbf{=}-symbool zeg je wat er in de variabele moet worden opgeslagen.
Je kunt dus ook de originele waarde terug veranderen. Dit is wat er gebeurt op
de 2de lijn. Eerst wordt \textbf{mijnVariabele} opgeteld bij 1 en wordt het
resultaat opgeslagen in \textbf{mijnVariabele} (mijnVariabele is nu 2).

Soorten waarden dat je kunt opslaan in een variabele:
\begin{itemize}
    \item \mintinline{javascript}|number| (nummer/getal): Vb 1,2,3, 1.234, -4,
        ...
    \item \mintinline{javascript}|string| (tekenreeks of tekst):
        \mintinline{javascript}|"Dit is een string"|
    \item \mintinline{javascript}|bool| (waar of niet waar): Kan slechts 2 waarden bevatten: \mintinline{javascript}|true, false|.
    \item \mintinline{javascript}|array| (lijst): Lijst van waarden: Vb: lijst
        van nummers: \mintinline{javascript}|[1,2,3,4]|.
    \item \mintinline{javascript}|object| (object): Alles wat je niet kunt
        beschrijver als een nummer of stuk tekst (Vb
        \mintinline{javascript}|document,window,console|)
    \item \mintinline{javascript}|null,undefined| (niets, ongedefinieerd):
        opslaan van dingen die je niet kunt opslaan of nog niet weet.
\end{itemize}

\subsubsection{Conditioneel code uitvoeren}%
\label{ssub:Conditioneel code uitvoeren}

Algemene structuur:
\begin{minted}{javascript}
if (testbool) {
    // uit to voeren als true
} else {
    // code uit te voeren als false
}
\end{minted}

Als 1 kleiner is dat 2, print dan naar de console:
\begin{minted}{javascript}
if (1 < 2) {
    console.log("1 is kleiner dan 2");
} else {
    console.log("1 is niet kleiner dan 2");
}
\end{minted}

Je hoeft else niet altijd toe te voegen (optioneel):
\begin{minted}{javascript}
if (1 < 2) {
    console.log("1 is kleiner dan 2");
}
console.log("Ik wordt altijd uitgevoerd");
\end{minted}

Je kunt ook met variabelen werken:
\begin{minted}{javascript}
let i = 1;
if (i < 2) {
    console.log("i is kleiner dan 2");
}
i = 3;
if (i < 2) {
    // Wordt niet uitgevoerd
    console.log("i is kleiner dan 2");
} else {
    console.log("i is niet kleiner dan 2");
}
\end{minted}

\subsubsection{Code herhalen}%
\label{ssub:Code herhalen}

Structuur for i:
\begin{minted}{javascript}
for (let i = 0; i < 10; i++) {
    console.log(i);
}
// Uitvoer:
// 0
// 1
// 2
// 3
// 4
// 5
// 6
// 7
// 8
// 9
\end{minted}

Structuur for of
\begin{minted}{javascript}
let mijnLijst = [1,2,3,4];
for (let nummer of mijnLijst) {
    console.log(nummer);
}
// Uitvoer:
// 1
// 2
// 3
// 4
\end{minted}

Structuur while: zolang de conditie waar is, blijf herhalen
\begin{minted}{javascript}
let i = 10;
while (i > 5) {
    console.log(i);
    i = i - 1;
}
// Uitvoer:
// 10
// 9
// 8
// 7
// 6
\end{minted}

\subsubsection{Groeperen van code}%
\label{ssub:Groeperen van code}

Met de vorige stukken is het nu al mogelijk om al heel complexe programma's te
schrijven. Dit is ook hoe de eerste computers werden geprogrammeerd. Dit leidde 
echter tot het probleem dat code niet hergebruikt kon worden of het moeilijk 
werd voor een mens om het programma logisch nog te begrijpen.

Als antwoord hierop, werden functies (\mintinline{javascript}|function|)
uitgevonden. Dit is een stuk code dat je een naam geeft, optioneel een aantal
parameters en optioneel ook een variabele kan teruggeven. Deze functie kun je
dan overal hergebruiken als je de naam kent.

Structuur van een functie:
\begin{minted}{javascript}
function naamFunctie(parameter1, parameter2) {
    // Doe iets cool met parameter1 en parameter2
    let ietsCool = 1 + 1;
    return ietsCool;
}

// gebruiken van je nieuwe coole functie:
naamFunctie(waarde1, waarde2);
\end{minted}
